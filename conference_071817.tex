\documentclass[conference]{IEEEtran}
\IEEEoverridecommandlockouts
% The preceding line is only needed to identify funding in the first footnote. If that is unneeded, please comment it out.
\usepackage{cite}
\usepackage{amsmath,amssymb,amsfonts}
\usepackage{algorithmic}
\usepackage{graphicx}
\usepackage{textcomp}
\usepackage{url}
\usepackage{multicol}

\usepackage[dvipsnames]{xcolor}
\newcommand{\todo}[1]{\textcolor{red}{\emph{Todo: #1}}}

\begin{document}

\title{Integrated C-ITS Reference Architecture* \\
  {\footnotesize \textsuperscript{*}Industrial Experience Report}
  \thanks{Funded by European Union Commission , 723311.}
}

\author{
  \IEEEauthorblockN{Priyanka Karkhanis,
    Mark van den Brand, Saurab Rajkarnikar}
  \IEEEauthorblockA{Eindhoven University of Technology\\
    Eindhoven, The Netherlands\\
    Email: p.d.karkhanis@tue.nl,
    M.G.J.v.d.Brand@tue.nl,
    s.rajkarnikar@tue.nl}
}

\maketitle


\begin{abstract}
C-ITS (Cooperative Intelligent Transport Systems) is an initiative to facilitate cooperative, connected and automated mobility.
It is based on the concept of System of Systems and promotes a new way of thinking for solving grand challenges where the interactions of technology, policy and economics are the primary drivers.

The C-ITS domain comprises widely spread systems like traffic management systems, traffic light controllers, or vehicle on-board units.
Such complex and heterogeneous systems have independent uses but demand a strategy to facilitate their convergence.

The main objective of C-ITS is to define an integrated architecture based on a number of existing C-ITS projects.
The architecture provides a way to standardize and a unifying modeling approach by means of a common language that can be reused by other organizations to guide their internal development processes.
The architecture and its concepts are based on the conceptual model of the ISO/IEC/IEEE 42010 \cite{iso42010} international standard for architecture descriptions of systems, System of Systems and software.
It defines architecture viewpoints for C-ITS systems and uses the concept of architecture perspectives for shaping these architecture viewpoints.
In this paper, we present the C-MobILE C-ITS reference architecture that is aimed at large scale deployment and demonstration with partner deployment sites across Europe. Also,  \todo{add more in abstract}.
\end{abstract}

\begin{IEEEkeywords}
C-ITS, ITS, Architecture Framework, transportation
\end{IEEEkeywords}


\section{Introduction}

The European Parliament in its directive 2010/40/EU \cite{ec} defines Intelligent Transport Systems (ITS) as "systems in which information and communication technologies are applied in the field of road transport, including infrastructure, vehicles and users, and in traffic management and mobility management, as well as for interfaces with other modes of transport."
ITS can be further described as systems which aim to make transportation safe and economical by combining data from the vehicles and other sensors on the roadway together with weather information. It began during the 1990s\cite{itsbegin} with projects in
\begin{itemize}
	\item the US (named Intelligent Vehicle Highway System \cite{ivhs})
	\item various countries in Europe (with the program Prometheus \cite{prometheus})
	\item Japan (with a research committee Road/Automobile Communication System \cite{racs})
\end{itemize}

Cooperative Intelligent Transport Systems (C-ITS) \cite{c-its} adds upon ITS by providing ways for connected vehicles to interact with other connected vehicles or any infrastructure such as the traffic light controller, roadway signals or roadside units. This interaction is where the term cooperatives comes from. In this scenario the vehicles can act as sensors as well.

The C-ITS domain covers not only the field of software- and systems engineering, but also traffic engineering, civil engineering, and information technology, which require a unified architecture for the C-ITS domain.

The C-MobILE project (Accelerating C-ITS Mobility Innovation and deployment in Europe) is an EU project that spans across eight C-ITS equipped deployment sites and regions with more than 37 participating institutes and companies.


\begin{figure}[ht!]
	\centering
	\includegraphics[width=0.4\textwidth]{deploymentsites}
	\caption{The C-ITS equipped deployment sites partnering with C-MobILE. 1. Newcastle, UK, 2. Eindhoven and Helmond (North Brabant), The Netherlands, 3. Bordeaux, France, 4. Vigo, Spain, 5. Bilbao, Spain, 6. Barcelona, Spain, 7. Region of Central Macedonia, Greece and 8. Copenhagen, Denmark}
	\label{fig:deployment sites}
\end{figure}	

The eight C-ITS equipped deployment sites already have C-ITS services through various projects that took place in the past.
However, many of these are not compatible with each other.
C-MobILE plans to become a common approach that ensures compatibility and become a basis for large scale deployment in Europe.
It is working with various public and private stakeholders while carrying out and developing cost effective business models particularly from the end user's perspective.

To help reach the project goals, an architecture capable of being deployed to the whole of Europe is needed.
The architecture definition process in C-MobILE has been defined to support the following sub-goals.

\begin{itemize}
  \item Analyse existing C-ITS architectures to provide common concepts and vocabulary.
  \item Identify a set of patterns that have been detected (or applied implicitly) during the analysis of existing C-ITS architectures and their implementations.
  \item Create a reference architecture that enables pan-European interoperability of C-ITS (concrete/implementation) architectures based on the generalization of existing C-ITS architectures.
  \item Define an implementation architecture specifying components and their relationships (interfaces) guided and constrained by the C-ITS reference architecture.
  \item Identify service-relevant parts of the architecture and define services based on the business analysis.
\end{itemize}

\section{Related Works}
There are various C-ITS projects under development with several already completed and deployed successfully. Of those projects, the following were similar reference architecture projects.

\begin{enumerate}
	\item Dutch C-ITS Reference Architecture (DITCM)\cite{ditcm}\cite{ditcmits}:\\
		This project focused mostly in developing a reference architecture for large scale C-ITS deployment in the Netherlands. It was build based on current and some finished C-ITS projects.
	\item CONVERGE\footnote{\label{converge}Converge. \url{https://converge-online.de/}}: \\
		This was a German funded project that developed an open platform for service providers with focus on Car2X (V2X or Vehicle to Vehicle/Infrastructure) Systems Network.
	\item COMPASS4D\footnote{\label{compass4d}Compass4d. \url{http://ertico.com/projects/compass4d/}.}:\\
	This was an EU funded project that worked with three C-ITS services such as Road Hazard Warning, Red Light Violation Warning and Energy Efficiency Intersection Service.
	\item NordicWay\footnote{\label{nordicway}Nordicway.\url{http://vejdirektoratet.dk/EN/roadsector/Nordicway/Pages/Default.aspx}.}:\\
	This project, as the name suggests, focused mostly on the Nordic countries (Finland, Sweden, Norway and Denmark) and is a pre-deployment pilot project for  C-ITS deployment.
	\item US-ITS (ARC-IT)\footnote{\label{arcit}Arc-it version 8.1. \url{https://local.iteris.com/arc-it/}}:\\
	The US-ITS or ARC-IT project is a large scale reference architecture that acts as building blocks for small scale regional C-ITS projects in various regions of the USA.

\end{enumerate}

\section{Problem Statement}
The reference architecture projects sought to develop a base for future C-ITS deployments. However, these approaches eventually led to various architectures that defined their own multidisciplinary approach toward their deployed strategy. They provided standard notations to help in merging these architectures which turned out to be regional. For example, the Dutch C-ITS Reference Architecture used standard protocols but only relating to the Netherlands and CONVERGE only related to Germany while the US-ITS used their own protocols. NordicWay and Compass4D used a standard protocol, however, they did not focus on many services thus rendering those project unsuitable for rest of Europe.

In addition, the partner deployment sites (\ref{fig:deployment sites}) have their own C-ITS implementations with their own C-ITS architecture, ad-hoc notations and differing categorizations.

Thus, there doesn't seem to be a standard notation for use in a large scale deployment. This demands a standardized approach to consolidate and integrate existing architectures, addressing concerns such as security and maintainability.

This is where the C-MobILE project comes in. It aims for a large scale demonstration across various deployment sites with an architecture that harmonizes existing technologies without changing the existing architectures. It also plans to be a building block or the base for future C-ITS implementations in other cities or regions.

\section{Methodology}
\begin{figure}[ht!]
	\centering
	\includegraphics[width=0.45\textwidth]{methodology}
	\caption{Developing a reference architecture for C-ITS by extracting and reverse engineering of existing architectures}
 	\label{methodology}
 	\centering
\end{figure}
To develop a common and compatible reference architecture, the following C-ITS projects were taken into consideration: (i) The Dutch C-ITS Reference Architecture (DITCM) \cite{ditcm}\cite{ditcmits}, (ii) CONVERGE\footnotemark[\ref{converge}], (iii) COMPASS4D\footnotemark[\ref{compass4d}], (iv) MOBiNET\footnote{\label{mobinet}MOBiNET. \url{http://www.mobinet.eu/}}, (v) NordicWay\footnotemark[\ref{nordicway}], (vi) SCOOP@F\footnote{SCOOP@F. \url{https://ec.europa.eu/inea/en/connecting-europe-facility/cef-transport/projects-by-country/multi-country/2014-eu-ta-0669-s}} and (vii) US-ITS (ARC-IT)\footnotemark[\ref{arcit}].

Besides these C-ITS architectures, we considered ITS implementations of the deployed sites involved with C-MobILE.
We applied a reverse architect approach by extracting the systems, protocols, networks, and technology details from these architectures manually (Fig.: \ref{methodology}).

To define the reference architecture Systems Modeling Language or SysML was proposed. SysML is a general purpose modeling language for engineering systems, and consist of structure diagrams, requirement diagrams and behavior diagrams. The architecture framework and a unified modelling approach can enable common language and will be reused for the next deployment projects. Furthermore, organizations can use the architecture to guide their internal development process as it reflects a common understanding of how the ITS landscape will evolve.

More info on methodology in \todo{give name of paper}.

\section{C-MobILE Reference Architecture}
\label{secCMobILEReferenceArchitecture}

The C-MobILE Reference Architecture focuses on an abstract level and uses a black box approach wherever possible.
It describes various systems at a high level in the form of models using SysML Block Definition Diagrams (BDD) and Internal Block Diagrams (IBD), providing base level information to architects. These block diagrams capture architecture viewpoints defined for C-MobILE. 
The C-MobILE project neither has the resources nor the intention to redefine all those standards.
Instead, at high level we highlight the common systems, their interfaces and protocols by considering various existing projects to ensure interoperability.

As a result of the architecture analysis and reverse architecting process, we have extracted the reference architecture from various existing architectures, which was consistent with the DITCM reference architecture \cite{ditcm}\cite{ditcmits}. As an illustration of the reference architecture, we discuss here the context and functional viewpoints by displaying the context model representation in Figure \ref{fig:contextviewpoint} and the functional model representation in Figure \ref{fig:functional}. The models presented are at an abstract level which will be further decomposed into an architecture that will be used for implementation.

The system structure is captured by categorizing into systems such as Central, Roadside, Vehicle and Traveler/VRU System with a Support System that supports all the other systems (Figures \ref{fig:contextviewpoint} and \ref{fig:functional}). Here, VRU means Vulnerable Road User such as a pedestrian or a cyclist. These systems are further decomposed into subsystems such as Traffic Management System (TMS) for the Central System (Figure \ref{fig:functional}).

\begin{figure}[ht!]
	\centering
	\includegraphics[width=0.45\textwidth]{context}
	\caption{Context Model Representation of the C-MobILE Reference Architecture}
	\label{fig:contextviewpoint}
	\centering
\end{figure}
The context viewpoint describes the relationships, dependencies, and interactions between the system and its environment (e.g. people,
systems and external entities) \cite{sysml}. The context view (Figure \ref{fig:contextviewpoint})conforms to the context viewpoint and helps system’s stakeholders (e.g. system/software architects, designers, developer and users) understand the system context.

The systems as mentioned before in terms of the context viewpoint is as below:

\begin{itemize}
	\item Support System: Comprises of sub-systems performing various tasks e.g.: governance, test and certification management, security and credentials management.
	\item Central System: Comprises of sub-systems to support connected vehicles, field and mobile devices. The sub-systems can be aggregated together or geographical or functionally distributed
	\item Roadside System: Comprises of sub-systems which covers the ITS infrastructure on or along physical road infrastructure, e.g.: roadside units, signal/lane control etc.
	\item Vehicle System: Comprises of sub-systems which are integrated within vehicle such on-board systems (advanced driver assistance / safety systems, navigation, remote data collection or information).
	\item Traveler/VRU System: Comprises of both personal devices (e.g.: mobile devices, navigations devices) and specific systems connected to vehicles of VRU’s (e.g.: tags).
\end{itemize}

\begin{figure}[ht!]
	\centering
	\includegraphics[width=0.45\textwidth]{functional}
	\caption{Functional Model Representation of the C-MobILE Reference Architecture}
	\label{fig:functional}
	\centering
\end{figure}

A functional viewpoint in the abstract level describes the system's runtime functional elements and primary interactions. The functional view (Figure \ref{fig:functional}) conforms to the functional viewpoint, helps the system's stakeholders understand the system structure, and has an impact on the system's quality properties.

As discussed previously, the systems were further decomposed into subsystems. The subsystems are depicted in the Functional model in Figure \ref{fig:functional} as functional elements with primary interaction with other subsystems. 

The subsystems for the Central System are the Traffic Management System (TMS: A functional back-office system of the responsible road operator to enforce legal actions on the roads), SP/DP/TIS BO (Generic Back-Office systems for Service Providers and Data Providers that collects and fuses date from service providers, vehicles and infrastructure), Service Provider Exchange System (SPES: An e-Market system for discovery and exchange of ITS services) and Communication Data Provider (A generic back-office system of a communication provider user for access at several communication systems from other BO systems).

The subsystems for the Roadside System are Roadside (Different types of existing systems such as a Traffic Light controller or substations) and Roadside Unit (A cooperative roadside communication system responsible for two-way communication functionality at a part of a road network).

The subsystems for the Vehicle System are On Board Unit (OBU: A sub-system attached to a car and needed for informing/advising the driver through a HMI), Remote Vehicle On Board Unit (An OBU of the other vehicle that is communication with the host vehicle) and Vehicle Electrical \& Electronic System (The parts of the vehicle such as in-car light sensors, speed sensors and actuators).

The subsystems for the Traveler/VRU System are Personal Information Devices (PID: Typically a smartphone or personal navigation device used by an end-user) and Vulnerable Road Users On Board Unit (An OBU attached to a VRU vehicle like bicycle or moped and need to advices the cyclists or the moped driver).

\section{Lessons Learned}
\todo{Lessons Learned}

\section{Conclusions and Further Work}

\todo{conclusion}

In this paper we present the reference architecture defined for the C-MobILE project.
%In the C-MobILE project we analysed existing C-ITS architectures, especially CONVERGE, MOBiNET, and Dutch C-ITS Reference Architecture, to define common concepts and vocabulary for the C-MobILE reference, concrete, and implementation architectures.
%We used the different architectures from the deployment sites as an input for defining a single homogeneous reference architecture, which will be further refined.
%We employed a reverse architect approach for manually extracting components, systems, and technological details.
The next step is to automate this approach, with the intention to provide benefits for system architects and stakeholders.


\section*{Acknowledgments}

The C-MobILE project is funded by the European Union's Horizon 2020 research and innovation programme under grant agreement No 723311.


\begin{thebibliography}{00}

    \bibitem{ec} European Commission. Directive 2010/40/eu of the european parliament and of the council.
    \bibitem{ecits} European Commission. Intelligent transport systems. \url{https://ec.europa.eu/transport/themes/its/c-its_en}.
    \bibitem{ditcm} Marcel van Sambeek, Frank Ophelders, Tjerk Bijlsmaand Borgert van der Kluit, Oktay Turetken, Rik Eshuis, Kostas Traganos, and Paul Grefen. Towards an architecture for cooperative-intelligent transport system (c-its) applications in the netherlands. Available at \url{ https://www.researchgate.net/publication/313580768_Towards_an_Architecture_for_Cooperative-Intelligent_Transport_System_C-ITS_Applications_in_the_Netherlands}.

	\bibitem{ditcmits} Dr Igor Passchier, Dr Ir Marcel van sambeek, Ir Joelle van den Broek and Drs Ir Paul Potters. The Dutch C-ITS Reference architecture. Paper number EU-TP0105. 11th ITS European Congress, Glasgow, Scotland, 6-9 June 2016.

    \bibitem{iso42010} ISO/IEC/IEEE Systems and software engineering -- Architecture description," in ISO/IEC/IEEE 42010:2011(E) (Revision of ISO/IEC 42010:2007 and IEEE Std 1471-2000) , vol., no., pp.1-46, Dec. 1 2011 doi: 10.1109/IEEESTD.2011.6129467
    
    \bibitem{archframework}Emery, D., and Rich H. Every architecture description needs a framework: Expressing architecture frameworks using ISO/IEC 42010. Software Architecture, 2009 \& European Conference on Software Architecture. WICSA/ECSA 2009. Joint Working IEEE/IFIP Conference on. IEEE, 2009.
    
    \bibitem{ITSCongress}Raul Ferrandez, Yanja Dajsuren, Priyanka Karkhanis, Manuel Fünfrocken, Marcos Pillado. C-MobILE C-ITS Reference Architecture. ITS World Congress 2018 (unpublished).
    \bibitem{c-its}A. Festag, "Cooperative intelligent transport systems standards in europe," in IEEE Communications Magazine, vol. 52, no. 12, pp. 166-172, December 2014.  doi: 10.1109/MCOM.2014.6979970
    
    \bibitem{itsbegin} A. Luis Osório, Hamideh Afsarmanesh, and Luis M. Camarinha-Matos. Towards a reference architecture
    for a collaborative intelligent transport system infrastructure. In Luis M. Camarinha-Matos, Xavier Boucher, and Hamideh Afsarmanesh, editors, Collaborative Networks for a Sustainable
    World, pages 469–477, Berlin, Heidelberg, 2010. Springer Berlin Heidelberg.
    
    \bibitem{ivhs} P.D. Heermann and D.L. Caskey. Intelligent vehicle highway system: Advanced public transportation
    systems. Mathematical and Computer Modelling, 22(4):445 – 453, 1995.
    
    \bibitem{prometheus} M. Williams. Prometheus-the european research programme for optimising the road transport
    system in europe. In IEE Colloquium on Driver Information, pages 1/1–1/9, Dec 1988.
    
    \bibitem{racs} K. Takada, Y. Tanaka, A. Igarashi, and D. Fujita. Road/automobile communication system
    (racs) and its economic effect. In Vehicle Navigation and Information Systems Conference,
    1989. Conference Record, pages A15–A21, Sept 1989.
    
    \bibitem{sysml} Rozanski, N. and Woods, E. Software systems architecture: working with stakeholders using viewpoints and perspectives. Addison-Wesley, 2011.
\end{thebibliography}


\end{document}
