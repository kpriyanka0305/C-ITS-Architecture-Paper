\documentclass[conference]{IEEEtran}
\IEEEoverridecommandlockouts
% The preceding line is only needed to identify funding in the first footnote. If that is unneeded, please comment it out.
\usepackage{cite}
\usepackage{amsmath,amssymb,amsfonts}
\usepackage{algorithmic}
\usepackage{graphicx}
\usepackage{textcomp}
\usepackage{url}
\usepackage{multicol}


\usepackage[dvipsnames]{xcolor}
\newcommand{\todo}[1]{\textcolor{red}{\emph{Todo: #1}}}

\begin{document}

\title{Defining C-ITS Reference Architecture* \\
  {\footnotesize \textsuperscript{*}Industrial Experience Report}
  \thanks{Funded by European Union Commission , 723311.}
}

\author{
  \IEEEauthorblockN{Priyanka Karkhanis,
    Mark van den Brand, Saurab Rajkarnikar}
  \IEEEauthorblockA{Eindhoven University of Technology\\
    Eindhoven, The Netherlands\\
    Email: p.d.karkhanis@tue.nl,
    M.G.J.v.d.Brand@tue.nl,
    s.rajkarnikar@tue.nl}
}

\maketitle


\begin{abstract}
C-ITS (Cooperative Intelligent Transport Systems) is an initiative to facilitate cooperative, connected and automated mobility.
It is based on the concept of System of Systems and promotes a new way of thinking for solving grand challenges where the interactions of technology, policy and economics are the primary drivers.

The C-ITS domain comprises widely spread systems like traffic management systems, traffic light controllers, or vehicle on-board units.
Such complex and heterogeneous systems have independent uses but demand a strategy to facilitate their convergence.

The main objective of C-ITS is to define an integrated architecture based on a number of existing C-ITS projects.
The architecture provides a way to standardize and a unifying modeling approach by means of a common language that can be reused by other organizations to guide their internal development processes.
The architecture and its concepts are based on the conceptual model of the ISO/IEC/IEEE 42010 \cite{iso42010} international standard for architecture descriptions of systems, System of Systems and software.
It defines architecture viewpoints for C-ITS systems and uses the concept of architecture perspectives for shaping these architecture viewpoints.
In this paper we present the methodology used for defining the C-MobILE reference architecture.
We demonstrate this by means of the C-MobILE reference architecture, which allows large scale demonstrations of integrations of C-ITS systems across Europe.
\end{abstract}

\begin{IEEEkeywords}
C-ITS, ITS, architecture framework, transportation
\end{IEEEkeywords}


\section{Introduction}

The European Parliament in its directive 2010/40/EU \cite{ec} defines Intelligent Transport Systems (ITS) as "systems in which information and communication technologies are applied in the field of road transport, including infrastructure, vehicles and users, and in traffic management and mobility management, as well as for interfaces with other modes of transport."
ITS can be further described as systems which aim to make transportation safe and economical by combining data from the vehicles and other sensors on the roadway together with weather information. It began during the 1990s\cite{itsbegin} with projects in
\begin{itemize}
	\item the US (named Intelligent Vehicle Highway System \cite{ivhs})
	\item various countries in Europe (with the program Prometheus \cite{prometheus})
	\item Japan (with a research committee Road/Automobile Communication System \cite{racs})
\end{itemize}

Cooperative Intelligent Transport Systems (C-ITS) \cite{c-its} adds upon ITS by providing ways for connected vehicles to interact with other connected vehicles or any infrastructure such as the traffic light controller, roadway signals or roadside units. This interaction is where the term cooperatives comes from. In this scenario the vehicles can act as sensors as well.

The C-ITS domain covers not only the field of software- and systems engineering, but also traffic engineering, civil engineering, and information technology, which require a unified architecture for the C-ITS domain.

It aims to make road transport safe and efficient while decreasing casualties and serious injuries on European roads.

The C-MobILE project (Accelerating C-ITS Mobility Innovation and deployment in Europe) is an EU project that spans across eight C-ITS equipped deployment sites and regions with more than 37 participating institutes and companies.

\begin{center}
\begin{figure}[ht!]
	\includegraphics[width=0.49\textwidth]{deploymentsites}
	\caption{The C-ITS equipped deployment sites partnering with C-MobILE. 1. Newcastle, UK, 2. Eindhoven and Helmond (North Brabant), The Netherlands, 3. Bordeaux, France, 4. Vigo, Spain, 5. Bilbao, Spain, 6. Barcelona, Spain, 7. Region of Central Macedonia, Greece and 8. Copenhagen, Denmark}
	\label{fig:deployment sites}
\end{figure}	
\end{center}

The eight C-ITS equipped deployment sites already have C-ITS services through various projects that took place in the past.
However, many of these are not compatible with each other.
C-MobILE plans to become a common approach that ensures compatibility and become a basis for large scale deployment in Europe.
It is working with various public and private stakeholders while carrying out and developing cost effective business models particularly from the end user's perspective.

To help reach the project goals, an architecture capable of being deployed to the whole of Europe is needed.
The architecture definition process in C-MobILE has been defined to support the following sub-goals.

\begin{itemize}
  \item Analyse existing C-ITS architectures to provide common concepts and vocabulary.
  \item Identify a set of patterns that have been detected (or applied implicitly) during the analysis of existing C-ITS architectures and their implementations.
  \item Create a C-ITS reference architecture that enables pan-European interoperability of C-ITS (concrete/implementation) architectures based on the generalization of existing C-ITS architectures.
  \item Define an implementation architecture specifying components and their relationships (interfaces) guided and constrained by the C-ITS reference architecture.
  \item Identify service-relevant parts of the architecture and define services based on the business analysis.
\end{itemize}


\section{Problem Statement}

The C-MobILE project aims for a large scale demonstration across various C-ITS equipped deployment sites. These deployment sites have defined their own ITS architecture, and their own multidisciplinary approach towards their deployed strategy. These architectures comprise of different informal design patterns.
There is no standard notation to help in merging these architectures into a silo-based architecture.
Silo-based means harmonizing existing technologies without changing them.
Without a common C-ITS architecture framework, different categorizations and ad-hoc notations have been used in existing C-ITS architectures.
This demands a standardized approach to consolidate and integrate existing architectures, addressing concerns such as security, availability, and maintainability.

%\begin{figure}[ht!]
%  \centering
%  \includegraphics[width=0.5\textwidth]{functional_vp}
%  \caption{Functional model capturing functional viewpoint of the C-ITS systems for C-MobILE.}
%  \label{functional}
%  \centering

%\end{figure}


\section{Methodology}

To develop a common and compatible reference architecture, the following C-ITS projects were taken into consideration: (i) The Dutch C-ITS Reference Architecture (DITCM) \cite{ditcm}, (ii) CONVERGE\footnote{Converge. \url{https://converge-online.de/}}, (iii) COMPASS4D\footnote{Compass4d. \url{http://ertico.com/projects/compass4d/}.}, (iv) MOBiNET\footnote{MOBiNET. \url{http://www.mobinet.eu/}.}, (v) NordicWay\footnote{Nordicway. \url{http://vejdirektoratet.dk/EN/roadsector/Nordicway/Pages/Default.aspx}.} and (vi) US-ITS (ARC-IT) \footnote{Arc-it version 8.1. \url{https://local.iteris.com/arc-it/}.}.

Besides these C-ITS architectures, we considered ITS implementations of the deployed sites involved with C-MobILE.
We applied a reverse architect approach by extracting the systems, protocols, networks, and technology details from these architectures manually (Fig.: \ref{methodology}).

To define the reference architecture Systems Modeling Language or SysML was proposed.

\begin{figure}[ht!]
	\centering
	\includegraphics[width=0.5\textwidth]{methodology}
	\caption{Developing a reference architecture for C-ITS by extracting and reverse engineering of existing architectures}
	\label{methodology}
	\centering
	
\end{figure}

A repository has been constructed consolidating all necessary and required information from existing architectures.
A thorough analysis was done to extract the commonalities, but not leaving behind their specific implementation details.
As a well-defined architecture framework is important for any architecture description \cite{archframework}, we defined an architecture framework for the C-ITS domain \cite{ITSCongress}.
Architecture frameworks facilitate communication and cooperation between different stakeholders while designing and constructing complex systems such as C-ITS.
Many different stakeholders with their interweaving concerns require a systematic approach for addressing the complexity of the lifecycle of the system.
To put architecture framework and architecture description concepts in context, we extend the conceptual model of the ISO/IEC/IEEE 42010\cite{iso42010} architecture framework.
The C-ITS architecture framework specifies stakeholders, their concerns, viewpoints, model kinds, and correspondence rules.
C-ITS architects can use our architecture framework to represent their C-ITS architecture, concrete implementation and deployed site architectures.

To design the C-MobILE architecture framework, the architecture process has been split into three parts:
The design phase, the refinement phase, and the details phase.

In the design phase, we create the reference architecture by analysing existing architectures, described in section \ref{secCMobILEReferenceArchitecture}.
In parallel, use-cases, business-cases, and requirements for the C-MobILE system are collected.
In the refinement phase, we use the reference architecture to create the medium-level architecture.
Furthermore, services, interfaces, and concepts are described to provide a guideline for the final phase.
In the details phase, we describe interfaces and concepts in detail to create a low-level implementation architecture.

We propose to use Systems Modelling Language (SysML) diagrams for architectural notations of the C-ITS architectures.
SysML is a general purpose modelling language for engineering systems, and consists of structure diagrams, requirement diagrams, and behavior diagrams.
The architecture framework and a unified modelling approach can enable common language and will be reused for the next deployment projects.
Furthermore, organizations can use the architecture to guide their internal development process as it reflects a common understanding of how the ITS landscape will evolve.


\section{C-MobILE Reference Architecture}
\label{secCMobILEReferenceArchitecture}

The C-MobILE Reference Architecture focuses on an abstract level and uses a black box approach wherever possible.
It describes various systems at a high level in the form of models using SysML.
SysML Block Definition Diagrams (BDD) and Internal Block Diagrams are used to represent the models for the reference architecture at an abstract level, providing base level information to architects.
However, due to the heterogeneous nature of such interfaces this will not be possible for all interfaces of the architecture.
For example, there exist several competing standards for roadside infrastructure to communicate with traffic management centers.
The C-MobILE project neither has the resources nor the intention to redefine all those standards.
Instead, at high level we highlight the common systems, their interfaces and protocols by considering various existing projects to ensure interoperability.

As a result of the architecture analysis and reverse architecting process, we have extracted the reference architecture from various existing architectures, which was consistent with the DITCM reference architecture. \todo{What is DITCM?}
The structure of a system is captured in functional structure models using BDDs by categorizing into systems and decomposing a system into subsystems.
A system defines the functionality and functional data flow interfaces between systems that are required to support a particular ITS application.
A functional model digram is shown in Figure \ref{functional} for the capturing functional viewpoint.



A functional viewpoint describes the system's runtime functional elements, their responsibilities, interfaces, and primary interactions.
The functional view conforms to the functional viewpoint, helps the system's stakeholders understand the system structure, and has an impact on the system's quality properties.
The system structure is captured in functional models using BDDs and by categorizing into systems and decomposing a system into subsystems. \todo{This is the same sentence as in the last paragraph}
A system defines the functionality and functional data flow interfaces between systems that are required to support a particular ITS application.
Information flows depict the exchange of information between subsystems.


\section{Conclusions and Further Work}

In this paper we present the methodology used for defining the C-MobILE reference architecture.
In the C-MobILE project we have analysed existing C-ITS architectures, especially CONVERGE, MOBiNET, and Dutch C-ITS Reference Architecture, to define common concepts and vocabulary for the C-MobILE reference, concrete, and implementation architectures.
We used the different architectures from the pilot sites as an input for defining a single homogeneous reference architecture, which will be further refined.
We employed a reverse architect approach for manually extracting components, systems, and technological details.
The next step is to automate this approach, with the intention to provide benefits for system architects and stakeholders.


\section*{Acknowledgments}

The C-MobILE project is funded by the European Union's Horizon 2020 research and innovation programme under grant agreement No 723311.


\begin{thebibliography}{00}

    \bibitem{ec} European Commission. Directive 2010/40/eu of the european parliament and of the council.
    \bibitem{ecits} European Commission. Intelligent transport systems. \url{https://ec.europa.eu/transport/themes/its/c-its_en}.
    \bibitem{ditcm} Marcel van Sambeek, Frank Ophelders, Tjerk Bijlsmaand Borgert van der Kluit, Oktay Turetken, Rik Eshuis, Kostas Traganos, and Paul Grefen. Towards an architecture for cooperative-intelligent transport system (c-its) applications in the netherlands. Available at \url{ https://www.researchgate.net/publication/313580768_Towards_an_Architecture_for_Cooperative-Intelligent_Transport_System_C-ITS_Applications_in_the_Netherlands}.

    \bibitem{iso42010} ISO/IEC/IEEE Systems and software engineering -- Architecture description," in ISO/IEC/IEEE 42010:2011(E) (Revision of ISO/IEC 42010:2007 and IEEE Std 1471-2000) , vol., no., pp.1-46, Dec. 1 2011 doi: 10.1109/IEEESTD.2011.6129467
    
    \bibitem{archframework}Emery, D., and Rich H. Every architecture description needs a framework: Expressing architecture frameworks using ISO/IEC 42010. Software Architecture, 2009 \& European Conference on Software Architecture. WICSA/ECSA 2009. Joint Working IEEE/IFIP Conference on. IEEE, 2009.
    
    \bibitem{ITSCongress}Raul Ferrandez, Yanja Dajsuren, Priyanka Karkhanis, Manuel Fünfrocken, Marcos Pillado. C-MobILE C-ITS Reference Architecture. ITS World Congress 2018 (unpublished).
    \bibitem{c-its}A. Festag, "Cooperative intelligent transport systems standards in europe," in IEEE Communications Magazine, vol. 52, no. 12, pp. 166-172, December 2014.  doi: 10.1109/MCOM.2014.6979970
    
    \bibitem{itsbegin} A. Luis Osório, Hamideh Afsarmanesh, and Luis M. Camarinha-Matos. Towards a reference architecture
    for a collaborative intelligent transport system infrastructure. In Luis M. Camarinha-Matos, Xavier Boucher, and Hamideh Afsarmanesh, editors, Collaborative Networks for a Sustainable
    World, pages 469–477, Berlin, Heidelberg, 2010. Springer Berlin Heidelberg.
    
    \bibitem{ivhs} P.D. Heermann and D.L. Caskey. Intelligent vehicle highway system: Advanced public transportation
    systems. Mathematical and Computer Modelling, 22(4):445 – 453, 1995.
    
    \bibitem{prometheus} M. Williams. Prometheus-the european research programme for optimising the road transport
    system in europe. In IEE Colloquium on Driver Information, pages 1/1–1/9, Dec 1988.
    
    \bibitem{racs} K. Takada, Y. Tanaka, A. Igarashi, and D. Fujita. Road/automobile communication system
    (racs) and its economic effect. In Vehicle Navigation and Information Systems Conference,
    1989. Conference Record, pages A15–A21, Sept 1989.
\end{thebibliography}


\end{document}
